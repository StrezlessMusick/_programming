\documentclass{article}
\usepackage[utf8]{inputenc}
\usepackage[russian]{babel}
\usepackage{listings}
\usepackage{fullpage}

\begin{document}

\begin{center}
\Large{Three(two) dimensional histogramms}
\end{center}

\section{Input}

\lstset{language=Fortran}
In file share/input.cfg:

%\begin{verbatim}
\begin{lstlisting}
! Multidimensional histogramms

&MultiDimHist
!		number of 3d/2d histogramms
  nmdh = 3,	
  
!		name of histo, which identify it for output
!		it not affect for any other
  mdh_name(1) = 'm34-y34',
!		"dimension" of histo; must be only 2 or 3
  mdh_dim(1) = 2,
!		flag for histo, which turns it on/off
!		and second its bin enable/dizable normalization values of bins
!		for its sizes
  mdh_flag(1) = 0,
!		names of observables;
!		number of it must be the same as mdh_dim;
!		each must be one of names of fixed or variable bin histogramms
!		except 'phis', which now names 'phistar'
!		; this names only select observable, and are not printed in output
  mdh_bname(1,1:2) = 'mtr' 'eta34',
!		number of bins for each observable
  mdh_nbins(1,1:2) = 9 4,
!	for aech observable:
!		borders of bins;
!		numbers of it must be on 1 more than according mdh_nbins
  mdh_bins1(1,1:10) = 20d0 30d0 50d0 70d0 90d0 110d0 130d0 150d0 200d0 300d0,
  mdh_bins2(1,1:5) = 0.00d0 0.25d0 0.5d0 0.75d0 1d0,
  mdh_bins3(1,1:2) = 0d0 0d0

  mdh_name(2) = 'm34-y34-pt34',
  mdh_dim(2) = 3,
  mdh_flag(2) = 0,
  mdh_bname(2,1:3) = 'm34' 'eta34' 'pt34',
  mdh_nbins(2,1:3) = 10 4 3,
  mdh_bins1(2,1:11) = 20d0 30d0 50d0 70d0 90d0 110d0 130d0 150d0 200d0 300d0 400d0,
  mdh_bins2(2,1:5) = 0.00d0 0.25d0 0.5d0 0.75d0 1d0,
  mdh_bins3(2,1:4) = 10d0 20d0 30d0 40d0

  mdh_name(3) = 'm34-test',
  mdh_dim(3) = 2,
  mdh_flag(3) = 1,
  mdh_bname(3,1:2) = 'm34' 'm34',
  mdh_nbins(3,1:2) = 13 1,
  mdh_bins1(3,1:14) = 20d0 30d0 50d0 70d0 90d0 110d0 130d0 150d0 200d0 300d0 400d0 500d0 1000d0 1500d0,
  mdh_bins2(3,1:2) = 50d0 8d3,
  mdh_bins3(3,1:2) = 0d0 0d0
/
%\end{verbatim}
\end{lstlisting}

\section{Output}

for each histogram In file output.txt : 
\begin{itemize}
\item its name (for 3d histogram value specified by [mdh\_name])

\item for each 3d/2d bin, in single line

\begin{itemize}
\item for first observable low 
\item and high border, 
\item for second observable low
\item  and high border, 
\item (if histogram is 3d) for third observable low 
\item and high border
\item bin value 
\item and its statistical error
\end{itemize}
\end{itemize}

\noindent 
as usually every histogram is written to a separate file

\end{document}
